An example of disaggregation calculation is given considering a source model
consisting of two sources (area and simple fault) belonging to two different
tectonic region types.

The calculation is defined with the following configuration file:

\begin{Verbatim}[frame=single, commandchars=\\\{\}, fontsize=\normalsize]
[general]
description = ...
calculation_mode = disaggregation
random_seed = 23

[geometry]
sites = 0.5 -0.5

[logic_tree]
number_of_logic_tree_samples = 0

[erf]
rupture_mesh_spacing = 2
width_of_mfd_bin = 0.1
area_source_discretization = 5.0

[site_params]
reference_vs30_type = measured
reference_vs30_value = 600.0
reference_depth_to_2pt5km_per_sec = 5.0
reference_depth_to_1pt0km_per_sec = 100.0

[calculation]
source_model_logic_tree_file = source_model_logic_tree.xml
gsim_logic_tree_file = gmpe_logic_tree.xml
investigation_time = 50.0
intensity_measure_types_and_levels = {"PGA": [...]}
truncation_level = 3
maximum_distance = 200.0

[disaggregation]
poes_disagg = 0.1
mag_bin_width = 1.0
distance_bin_width = 10.0
coordinate_bin_width = 0.2
num_epsilon_bins = 3

[output]
export_dir = ...
\end{Verbatim}

Disaggregation matrices are computed for a single site (located between the
two sources) for a ground motion value corresponding to a probability value
equal to 0.1 (\texttt{poes\_\-disagg = 0.1}). Magnitude values are classified
in one magnitude unit bins (\texttt{mag\_\-bin\_\-width = 1.0}), distances in
bins of 10 km (\texttt{distance\_\-bin\_\-width = 10.0}), coordinates in bins
of 0.2 degrees (\texttt{coordinate\_\-bin\_\-width = 0.2}). 3 epsilons bins
are considered (\texttt{num\_\-epsilon\_\-bins = 3}).
