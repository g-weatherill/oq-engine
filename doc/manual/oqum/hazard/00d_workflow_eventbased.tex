Input data for the Event-Based PSHA - as in the case of the Classical
\gls{acr:psha} calculator - consists of a PSHA Input Model and a set of
calculation settings.

The main calculators used to perform this analysis are:

\begin{enumerate}

	\item \emph{Logic Tree Processor}

	The Logic Tree Processor works in the same way described in  the
	description of the Classical \gls{acr:psha} workflow  (see
	Section~\ref{subsec:classical_psha} at
	page~\pageref{subsec:classical_psha}).

	\item \emph{Earthquake Rupture Forecast Calculator}

	The Earthquake Rupture Forecast Calculator was already  introduced in the
	description of the PSHA workflow (see Section~\ref{subsec:classical_psha}
	at page~\pageref{subsec:classical_psha}).

	\item \emph{Stochastic Event Set Calculator}

	The Stochastic Event Set Calculator generates a collection of stochastic
	event sets by sampling the ruptures contained in the ERF according to
	their probability of occurrence.

	A Stochastic Event Set (SES) thus represents a potential realisation of
	the seismicity (i.e. a list of ruptures) produced by the set of seismic
	sources considered in the analysis over the time span fixed for the
	calculation of hazard.

	\item \emph{Ground Motion Field Calculator}

	The Ground Motion Field Calculator computes for each event contained in a
	Stochastic Event Set a realization of the geographic distribution of the
	shaking by taking into account the aleatory uncertainties in the ground-
	motion model. Eventually, the Ground Motion Field calculator can consider
	the spatial correlation of the ground-motion during the generation of the
	\gls{acr:gmf}.

	\item \emph{Event-based PSHA Calculator}

	The event-based PSHA calculator takes a (large) set of ground-motion
	fields representative of the possible shaking scenarios that the
	investigated area can experience over a (long) time span and for each
	site computes the corresponding hazard curve.

	This procedure is computationally intensive and is not recommended for
	investigating the hazard over large areas.

\end{enumerate}