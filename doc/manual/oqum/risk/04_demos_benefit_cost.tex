The loss exceedance curves used within this demo are produced using the
Classical Probabilistic Risk calculator. Thus, the process to produce the
seismic hazard curves described in Section~\ref{sec:demos_classical_risk} can
be employed here. Then, the risk calculations can be initiated using the
following command:

\begin{minted}[fontsize=\footnotesize,frame=single,bgcolor=lightgray]{shell-session}
user@ubuntu:~\$ oq engine --run job_risk.ini --hc 8971
\end{minted}

Alternatively, the hazard and risk jobs can be run sequentially using:

\begin{minted}[fontsize=\footnotesize,frame=single,bgcolor=lightgray]{shell-session}
user@ubuntu:~\$ oq engine --run job_hazard.ini,job_risk.ini
\end{minted}

which should produce the following output:

\begin{minted}[fontsize=\footnotesize,frame=single,bgcolor=lightgray]{shell-session}
Calculation 8976 completed in 14 seconds. Results:
  id | name
9087 | bcr-rlzs
\end{minted}
